\documentclass[12pt, letterpaper]{article}

\usepackage{amsmath}

\title{Fisher-Kolmogorov equations for neurodegenerative diseases}
\author{Andrea Boin, Giacomo Pauletti, Lorenzo Pettenuzzo}
\date{}

\begin{document}
\maketitle
\pagebreak

\tableofcontents
\pagebreak

% TODO: paper citation
\section{Introduction}
The objective of this project is to apply various numerical methods to solve the Fisher-Kolmogorov equations to reproduce the results of the paper \cite{diffusion-paper}. The Fisher-Kolmogorov equations can be used to effectively model the spread of misfolded proteins in the brain, a process associated with numerous neurodegenerative diseases.

\subsection{Fisher-Kolmogorov equation}
\[
\begin{cases}
\displaystyle \frac{\partial c}{\partial t} - \nabla \cdot (D \nabla c) - \alpha c(1 - c) = 0 & \text{in } \Omega\\
\displaystyle D \nabla c \cdot \mathbf{n} = 0 & \text{on } \partial \Omega\\
c(t=0)=c_0 & \text{in } \Omega
\end{cases}
\]
$c$: concentration of the misfolded protein in a zone of the brain $(0\leq c\leq1)$\\
$\alpha$: constant of concentration growth\\
$D$: diffusion coefficient of the misfolded protein.\\
It can be isotropic (a scalar) or anisotropic (a square matrix).\\
In case of anisotropic coefficient the term can be computed as:
$$\underbar{D}=d^{\text{ext}}\underbar{I}+d^{\text{axn}}(\mathbf{n}\otimes\mathbf{n})$$
where $d^{\text{ext}}$ is the extracellular diffusion term, $d^{\text{axn}}$ is the axonal diffusion term and $\mathbf{n}$ the direction of axonal diffusion.\\
Usually extracellular diffusion is slower than axonal diffusion: $d^{\text{ext}}<d^{\text{axn}}$.\\
The Fisher-Kolmogorov equation is a \textbf{diffusion-reaction} equation with a nonlinear forcing term that can be used to model population growth. In this case it is used to model the spreading of proteins in the brain.\\
The interested problem is a \textbf{nonlinear parabolic PDE} with \textbf{Neumann boundary conditions}. 
\pagebreak

\subsection{Mesh}
The mesh we used for the simulation is a 3D representation of a hemisphere of the human brain with 21211 points and 42450 cells.\\
To process the mesh with our software, we did convert the format from \textit{.stl} to \textit{.msh} using \textbf{GMSH} with the following procedure:
\begin{enumerate}
    \item Import the mesh (\textit{.stl}) in GMSH
    \item From the left menu, select "geometry $\rightarrow$ add $\rightarrow$ volume"
    \item Save the new generated \textit{.geo} file
    \item Define the 3D mesh: "mesh $\rightarrow$ define $\rightarrow$ 3D"
    \item Export the file as \textit{.msh}: "file $\rightarrow$ export $\rightarrow$ msh"
\end{enumerate}

\section{Methods}

The weak formulation of the problem is:

$$\int_\Omega\frac{\delta c}{\delta t}vd\Omega+\int_\Omega D\nabla c\nabla vd\Omega-\int_\Omega\alpha c(1-c)vd\Omega=0$$\\
We used two methods to discretize the Fisher-Kolmogorov equations:
\begin{itemize}
    \item A \textbf{linearly implicit} scheme in which the linear terms have been treated implicitly while the nonlinear terms explicitly to get rid of nonlinarities.
    \item A \textbf{fully implicit} scheme in which all terms in the equation have been treated implicitly, and then the nonlinear parts have been solved with the Newton method.
\end{itemize}

%\subsection{1D implementation}
%\subsubsection{Stability and Accuracy}
%\subsubsection{Algorithm}

\subsection{Linearly implicit}
\subsubsection{Stability and Accuracy}

\subsection{Fully implicit}
\subsubsection{Stability and Accuracy}

Stability and accuracy analysis of the implemented methods.

\section{Results and algorithmic comparation}
Results of the project for all the implemented algorithms and graphs.

\begin{thebibliography}{9}
    \bibitem{diffusion-paper}
    J. Weickenmeier, M. Jucker, A. Goriely, and E. Kuhl. A physics-based model explains the prion-like features of neurodegeneration in Alzheimer’s disease, Parkinson’s disease, and amyotrophic lateral sclerosis. Journal of the Mechanics and
    Physics of Solids, 124:264–281, 2019.
\end{thebibliography}

\end{document}