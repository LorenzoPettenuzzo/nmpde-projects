\documentclass[12pt, letterpaper]{article}

\usepackage{amsmath}
\usepackage{amsfonts}


\title{Fisher-Kolmogorov equations for neurodegenerative diseases}
\author{Andrea Boin, Giacomo Pauletti, Lorenzo Pettenuzzo}
\date{}

\begin{document}
\maketitle
\pagebreak

\tableofcontents
\pagebreak

\section{Introduction}
The objective of this project is to apply various numerical methods to solve the Fisher-Kolmogorov equations to reproduce the results of the paper \cite{diffusion-paper}. The Fisher-Kolmogorov equations can be used to effectively model the spread of misfolded proteins in the brain, a process associated with numerous neurodegenerative diseases.

\subsection{Fisher-Kolmogorov equation}
\[
\begin{cases}
\displaystyle \frac{\partial c}{\partial t} - \nabla \cdot (D \nabla c) - \alpha c(1 - c) = 0 & \text{in } \Omega\\
\displaystyle D \nabla c \cdot \mathbf{n} = 0 & \text{on } \partial \Omega\\
c(t=0)=c_0 & \text{in } \Omega
\end{cases}
\]
\begin{itemize}
    \item $c$: concentration of the misfolded protein in a region of the brain $(0\leq c\leq1)$
    \item $\alpha$: constant of concentration growth
    \item $D$: diffusion coefficient of the misfolded protein.\\
It can be isotropic (a scalar) or anisotropic (a square matrix).\\
In case of anisotropic coefficient the term can be computed as:

$$\underbar{D}=d^{\text{ext}}\underbar{I}+d^{\text{axn}}(\mathbf{n}\otimes\mathbf{n})$$

where $d^{\text{ext}}$ is the extracellular diffusion term, $d^{\text{axn}}$ is the axonal diffusion term and $\mathbf{n}$ the direction of axonal diffusion.

Usually extracellular diffusion is slower than axonal diffusion: $d^{\text{ext}}<d^{\text{axn}}$.

The Fisher-Kolmogorov equation is a \textbf{diffusion-reaction} equation with a nonlinear forcing term that can be used to model population growth. In this case it is used to model the spreading of proteins in the brain.

The interested problem is a \textbf{nonlinear parabolic PDE} with \textbf{Neumann boundary conditions}.
\end{itemize} 
\pagebreak

\subsection{Mesh}
The mesh we used for the simulation is a 3D representation of a hemisphere of the human brain with 21211 points and 42450 cells.\\
To process the mesh with our software, we did convert the format from \textit{.stl} to \textit{.msh} using \textbf{GMSH} with the following procedure:
\begin{enumerate}
    \item Import the mesh (\textit{.stl}) in GMSH
    \item From the left menu, select "geometry $\rightarrow$ add $\rightarrow$ volume"
    \item Save the new generated \textit{.geo} file
    \item Define the 3D mesh: "mesh $\rightarrow$ define $\rightarrow$ 3D"
    \item Export the file as \textit{.msh}: "file $\rightarrow$ export $\rightarrow$ msh"
\end{enumerate}

% TODO: consideration about the limits of the mesh (no axons, ...)

\subsection{Weak formulation and semi-discretized formulation}
By choosing $V=H^1=\{v\in L^2|\nabla v\in L^2\}$ and considering a time domain $(0, T)$, the weak formulation of the problem is:

\vspace{1em}
\noindent Find $c(t)\in V$ such that $\forall v\in V$ and $\forall t\in(0,T)$:
$$\begin{cases}\int_\Omega\frac{\delta c}{\delta t}vd\Omega+\int_\Omega D\nabla c\nabla vd\Omega-\int_\Omega\alpha c(1-c)vd\Omega=0\\c(t=0)=c_0\end{cases}$$

\vspace{1em}
\noindent By renaming:
\begin{itemize}
    \item $a(c,v)=\int_\Omega D\nabla c\nabla vd\Omega$
    \item $n(c,v)=-\int_\Omega\alpha c(1-c)vd\Omega$
\end{itemize}

\noindent By introducing a triangulation $T_h=\{K|\Omega=\bigcup K\}$ of the domain $\Omega$ and defining with it a polynomial space $$X_h=\{v_h\in C^0(\bar\Omega)|v_{h|k}\in\mathbb{P}^r(K),\forall K\in T_h\}$$ we can obtain the discrete space $V_h=V\cap X_h$ for our discrete formulation.

\noindent The semi-discrete formulation can then be written as:

\vspace{1em}
\noindent
Find $c_h\in V_h$ such that, $\forall v_h\in V_h$ and $\forall t\in(0,T)$:
$$\int_\Omega\frac{\delta c_h}{\delta t}v_hd\Omega+a(c_h,v_h)+n(c_h,v_h)=0$$
$$c_h(t=0)=c_{h,0}$$

\section{Methods}

We studied the problem with 3 methods and implemented 2 of them algorithmically:
\begin{itemize}
    \item An \textbf{explicit} scheme in which all terms have been treated explicitly to handle the nonlinear part of the model.
    \item A \textbf{mixed explicit/implicit} scheme in which the linear terms have been treated implicitly while the nonlinear terms explicitly to get rid of nonlinarities.
    \item An \textbf{implicit} scheme in which all terms in the equation have been treated implicitly, and then the nonlinear parts have been solved with the Newton method.
\end{itemize}

\noindent To obtain a full discretization of the problem we need to partition the time domain in $N$ partitions of size $\Delta t$, obtaining $(0, T)=(0,N\Delta t)=\bigcup_{n=1}^N(t^n, t^{n+1}]$ where $t^{n+1}-t^n=\Delta t$, $t^0=0$ and $t^N=T$. We can then use an upper-index notation to identify time dependent elements: $c^n = c(t^n)$.

\subsection{Explicit scheme}

The fully discrete formulation for the \textbf{explicit} scheme becomes:

\vspace{1em}
\noindent
Find $c_h(t)\in V_h$ such that, $\forall v_h\in V_h$, $c_h^0=c_{h,0}$ and $\forall n\in\{0, N\}$:
$$\int_\Omega\frac{c_h^{n+1}-c_h^n}{\Delta t}v_hd\Omega+a(c_h^n,v_h)+n(c_h^n,v_h)=0$$

\noindent By introducing a basis $\{\phi_i\}$ for the space $V_h$ the problem can be written as:

\vspace{1em}
\noindent
Find $c_h(t)\in V_h$ such that $c_h^0=c_{h,0}$ and $\forall n\in\{0, N\}$:
$$Mc^{n+1}=F^n$$
where the \textbf{mass matrix} can be computed as:
$$M_{ij}=\frac1{\Delta t}\langle\phi_j,\phi_i\rangle$$
and the \textbf{forcing term} is:
$$F_i^n=\frac1{\Delta t}\langle c_{h,j}^n,\phi_i\rangle-a(c_{h,j}^n,\phi_i)-n(c_{h,j}^n,\phi_i)$$

\subsubsection{Stability and Accuracy}
The accuracy is $O(\Delta t)$ for time and $O(h^2)$ for space.

% TODO: check values correctness
\noindent The stability condition of the explicit scheme is: $\Delta t\leq\min(\frac{h^2}{2D}, \frac2\alpha)$. The first term acts as a bottleneck for the method. With our values for example ($h=1[cm], D=1.5[cm/year], \alpha=0.5[1/year]$), $\Delta t\leq\min(\frac13, 4)=\frac13$. The following methods allow for a larger choice of $\Delta t$ and are generally faster so we decided to implement them.

\subsection{Mixed explicit/implicit scheme}
The full discretization for the \textbf{mixed explicit/implicit} scheme is:

\vspace{1em}
\noindent
Find $c_h\in V_h$ such that, $\forall v_h\in V_h$ and $c_h(t=0)=c_{h,0}$:
$$\int_\Omega\frac{c_h^{n+1}-c_h^n}{\Delta t}v_hd\Omega+a(c_h^{n+1},v_h)+n(c_h^n,v_h)=f(v_h)$$

\noindent The problem can be rewritten, by introducing a basis $\{\phi_i\}$ for $V_h$ as:

\vspace{1em}
\noindent
Find $c_h(t)\in V_h$ such that $c_h^0=c_{h,0}$ and $\forall n\in\{0, N\}$:
$$Mc^{n+1}=F^n$$
where the \textbf{mass matrix} can be computed as:
$$M_{ij}=\frac1{\Delta t}\langle\phi_j,\phi_i\rangle+a(\phi_j,\phi_i)$$
and the \textbf{forcing term} is:
$$F_i^n=\frac1{\Delta t}\langle c_{h,j}^n,\phi_i\rangle-n(c_{h,j}^n,\phi_i)$$

\subsubsection{Algorithm}

\subsubsection{Stability and Accuracy}
The accuracy for this method is the same as the explicit one: $O(\Delta t)$ for time and $O(h^2)$ for space.

\noindent The stability condition though is better: $\Delta t\leq\frac2\alpha=4$ allowing for a larger choice of $\Delta t$ and a quicker convergence.

\subsection{Implicit scheme}
\subsubsection{Stability and Accuracy}

Stability and accuracy analysis of the implemented methods.

\section{Results and algorithmic comparation}
Results of the project for all the implemented algorithms and graphs.
% TODO: graph about explosion of disease confirming real evidence
% TODO: say some proteins might act at different percentages of concentration
% TODO: parallelism

\begin{thebibliography}{9}
    \bibitem{diffusion-paper}
    J. Weickenmeier, M. Jucker, A. Goriely, and E. Kuhl. A physics-based model explains the prion-like features of neurodegeneration in Alzheimer’s disease, Parkinson’s disease, and amyotrophic lateral sclerosis. Journal of the Mechanics and
    Physics of Solids, 124:264–281, 2019.
\end{thebibliography}

\end{document}