\documentclass[12pt, letterpaper]{article}

\usepackage{amsmath}

\title{Fisher-Kolmogorov equations for neurodegenerative diseases}
\author{Andrea Boin, Giacomo Pauletti, Lorenzo Pettenuzzo}
\date{}

\begin{document}
\maketitle
\pagebreak

\tableofcontents
\pagebreak

\section{Introduction}
The objective of this project is to apply various numerical methods to solve the Fisher-Kolmogorov equations, which are used to model the spread of misfolded proteins in the brain, a process associated with numerous neurodegenerative diseases.

\subsection{Fisher-Kolmogorov equation}
\[
\begin{cases}
\displaystyle \frac{\partial c}{\partial t} - \nabla \cdot (D \nabla c) - \alpha c(1 - c) = 0 & \text{in } \Omega\\
\displaystyle D \nabla c \cdot \mathbf{n} = 0 & \text{on } \partial \Omega\\
c(t=0)=c_0 & \text{in } \Omega
\end{cases}
\]
$c$: concentration of the misfolded protein in a zone of the brain $(0\leq c\leq1)$\\
$\alpha$: constant of concentration growth\\
$D$: diffusion coefficient of the misfolded protein.\\
It can be isotropic (a scalar) or anisotropic (a square matrix).\\
In case of anisotropic coefficient the term can be computed as:
$$\underbar{D}=d^{\text{ext}}\underbar{I}+d^{\text{axn}}(\mathbf{n}\otimes\mathbf{n})$$
where $d^{\text{ext}}$ is the extracellular diffusion term, $d^{\text{axn}}$ is the axonal diffusion term and $\mathbf{n}$ the direction of axonal diffusion. Usually extracellular diffusion is slower than axonal diffusion: $d^{\text{ext}}<d^{\text{axn}}$.\\

The Fisher-Kolmogorov equation is a \textbf{diffusion-reaction} equation with a nonlinear forcing term that can be used to model population growth. In this case it is used to model the spreading of proteins in the brain. 

\subsection{Mesh}
The mesh we used is a 3D representation of the human brain.
\pagebreak

\section{Methods}
List and description of the implemented methods with numerical analysis.

\subsection{1D implementation}
\subsubsection{Stability and Accuracy}
\subsubsection{Algorithm}

\subsection{Fully explicit}
\subsubsection{Stability and Accuracy}

\subsection{Semi-implicit}
\subsubsection{Stability and Accuracy}

\subsection{Fully implicit}
\subsubsection{Stability and Accuracy}

Stability and accuracy analysis of the implemented methods.

\section{Results and algorithmic comparation}
Results of the project for all the implemented algorithms and graphs.

\end{document}